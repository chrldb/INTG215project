\section{Background \& Context}
% SECTION 2

Over the past decade, the Internet of Things (IoT) witnessed dramatic expansion in scope, connectivity, and impact. IoT evolved from basic experimental concepts into an ambitious technology present in industries, cities, and homes, with promised revolutions on electronic equipment systems. Looking ahead, IoT is predicted to bring even greater utility to housing electronics, transforming smart homes from the fundamental perspectives with singular benefits including real-time automation, energy optimization, and predictive maintenance. 

By 2024, the number of connected IoT devices will have sharply increased to over 18.5 billion, marked as a symbol of significant market evolution, technological advancements, and enhanced interoperability of IoT devices. Adoption of new wireless technologies such as LTE-M, NB-IoT, and the introduction of 5G RedCap enabled affordable, low-latency connections suitable for a vast array of devices globally, which provides faster and more reliable communication for IoT and electronic ecosystems. Besides the improvements of connectivity, the proliferation of sensors and embedded systems has also expanded IoT’s reach, while the rapid evolution of Artificial Intelligence (AI) further empowered these devices to become smarter and more autonomous. 

Beside the industrial application that’s already fully demonstrated, IoT devices are predicted to be utilized among housing electronics, forming the foundational infrastructure for future smart homes. The interconnection of various home elements and equipment including lighting, heating, ventilation, air conditioning and security systems are enabled by IoT technologies. It provides the promised potential in improving enhanced energy efficiency, convenient automation, and improved living comfort by the employment of intelligent sensors combined with machine learning, allowing predictive maintenance and adaptive environmental control. The implementation of IoT devices for housing allows homeowners to remotely monitor and manage devices through automated systems. 

Furthermore, ongoing standardization efforts and integration with renewable energy systems such as solar panels and heat pumps provides further possibilities regarding the integration between Internet of Things (IoT) with residential electronics. All provide a sustainable and smarter resource management for housing electronics. 

In summary, significant progress was made within the field of IoT devices over the past decade, with the notable improvements in the device connectivity, technology complexity, and moreover the standardization. The foundation for IoT to revolutionize residential electronics, while enabling the creation of a highly interconnected, smart, and energy efficient housing environment is obtained. This development has led to a promised convenient and controllable housing network, which contributes to sustainable living and more responsive home management. It has highlighted the crucial role of the IoT within future residential technologies and frameworks. 
